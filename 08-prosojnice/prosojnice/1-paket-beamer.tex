%  Naslov prosojnice lahko naredimo tudi z dodatnim parametrom okolja `frame`.
\begin{frame}{Posebnosti prosojnic}
	% Naloga 2.3.1:
	% Dodajte ukaze, ki bodo poskrbeli, da se bo prosojnica odkrivala postopoma,
	% tako kot v datoteki prosojnice-resitev.pdf
	\begin{itemize}
		\item<1-> Za prosojnice je značilna uporaba okolja \texttt{frame},
		\item<2-> s katerim definiramo posamezno prosojnico,
		\item<3-> postopno odkrivanje prosojnic,
		\item<4-> ter nekateri drugi ukazi, ki jih najdemo v paketu \texttt{beamer}.
	\end{itemize}
	%
	\begin{exampleblock}{Primer}
		Verjetno ste že opazili, da za naslovno prosojnico niste uporabili
		ukaza \texttt{maketitle}, ampak ukaz \texttt{titlepage}.
	\end{exampleblock}
\end{frame}

\begin{frame}{Poudarjeni bloki}
	% Naloga 2.3.2:
	% Oblikujte poudarjena bloka z opombo in opozorilom.
	\begin{alertblock}{Opozorilo}
		To je opozorilni blok.
	\end{alertblock}
	\begin{exampleblock}{Opomba}
		To je blok z opombo.
	\end{exampleblock}
		Drugi parameter (za naslov) je lahko prazen. 

\end{frame}

\begin{frame}{Tudi v predstavitvah lahko pišemo izreke in dokaze}
	% Naloga 2.3.2:
	% Oblikujte okolje itemize, tako da se bo njegova vsebina postopoma odkrivala.
	% Ne smete uporabiti ukaza `pause'.
	% Beseda `največje' naj bo poudarjena šele na četrti podprosojnici.
	\begin{itemize}
		\item<1-> Naj bo $p$ \uncover<4->{\alert{največje}} praštevilo.
		\item<2-> Naj bo $q$ produkt števil $1$, $2$, \ldots, $p$.
		\item<3-> Število $q+1$ ni deljivo z nobenim praštevilom, torej je $q+1$ praštevilo.
		\item<4-> To je protislovje, saj je $q+1>p$. \qedhere
	\end{itemize}
	\begin{izrek}
	   $Praštevil je neskončno mnogo.$
	\end{izrek}
	\begin{proof}
	   Dokaz je preprost.	
	\end{proof}
 \end{frame}
 
