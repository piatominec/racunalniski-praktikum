\documentclass{beamer}
\usepackage{pgfpages}
% Naloga 1.3.1: Za dokument uporabite razred `beamer'.
% Ne dodajajte nastavitve za velikost pisave, kot je bila v datoteki `5-prosojnice.tex`.

% Naloga 1.3.2: vključite paket `predavanja'.

% Naloga 1.3.3: definirajte okolji `definicija' in `izrek'.
% Namig: z iskanjem po datotekah (Ctrl+Shift+F oz. Cmd+Shift+F) 
% poiščite niz `{definicija}' ali niz `{izrek}'.

\usepackage{amsmath}
\usepackage{amssymb}
\newtheorem{izrek}{Izrek}
\newtheorem{definicija}{Definicija}

\begin{document}
    
% Naloga 1.3.4: pripravite naslovno stran z vsebino:
\title{Matematični izrazi in uporaba paketa \texttt{beamer}}
\subtitle{\emph{Matematičnih} nalog ni treba reševati!}
\institute{Fakulteta za matematiko in fiziko}   
\date{}

% - naslov: Matematični izrazi in uporaba paketa \texttt{beamer}
% - podnaslov: \emph{Matematičnih} nalog ni treba reševati!
% - inštitut: Fakulteta za matematiko in fiziko
% - datum: naj se ne izpiše; to dosežete z ukazom \date{}.
% Zgornje podatke nastavite z ukazi kot v dokumentih razreda `article`.
% Več o tem, kako se naredi naslovno stran, si preberite na naslovu na naslovu:
% https://www.overleaf.com/learn/latex/Beamer
% To stran preberite do vključno razdelka "Creating a table of contents".
% Ukaz `\titlepage` deluje podobno kot ukaz `\maketitle`, ki ste ga že srečali.


% Naloga 1.3.5: pripravite kazalo vsebine.
% 1. Naslov prosojnice, s kazalom vsebine naj bo "Kratek pregled"
% 2. S pomožnim parametrom `pausesections' (v oglatih oklepajih) 
%    določite, da naj se kazalo vsebine odkriva postopoma.
%    Poglejte, kako deluje ta ukaz.
% 3. Ker ni videti preveč lepo, pomožni parameter zakomentirajte.

% \begin{frame}[pausesections]
%     \frametitle{Kratek pregled}
%     \tableofcontents
% \end{frame}

\section{Paket \texttt{beamer}}

\section{Paketa \texttt{amsmath} in \texttt{amsfonts}}

\section[Matematika, 1. del\\\large{Analiza, logika, množice}]{Matematika, 1. del}

\section{Stolpci in slike}

\section{Paket \texttt{beamer} in tabele}

\section[Matematika, 2. del\\\large{Zaporedja, algebra, grupe}]{Matematika, 2. del}

\end{document}
%  Naslov prosojnice lahko naredimo tudi z dodatnim parametrom okolja `frame`.
\begin{frame}{Posebnosti prosojnic}
	% Naloga 2.3.1:
	% Dodajte ukaze, ki bodo poskrbeli, da se bo prosojnica odkrivala postopoma,
	% tako kot v datoteki prosojnice-resitev.pdf
	\begin{itemize}
		\item<1-> Za prosojnice je značilna uporaba okolja \texttt{frame},
		\item<2-> s katerim definiramo posamezno prosojnico,
		\item<3-> postopno odkrivanje prosojnic,
		\item<4-> ter nekateri drugi ukazi, ki jih najdemo v paketu \texttt{beamer}.
	\end{itemize}
	%
	\begin{exampleblock}{Primer}
		Verjetno ste že opazili, da za naslovno prosojnico niste uporabili
		ukaza \texttt{maketitle}, ampak ukaz \texttt{titlepage}.
	\end{exampleblock}
\end{frame}

\begin{frame}{Poudarjeni bloki}
	% Naloga 2.3.2:
	% Oblikujte poudarjena bloka z opombo in opozorilom.
	\begin{alertblock}{Opozorilo}
		To je opozorilni blok.
	\end{alertblock}
	\begin{exampleblock}{Opomba}
		To je blok z opombo.
	\end{exampleblock}
		Drugi parameter (za naslov) je lahko prazen. 

\end{frame}

\begin{frame}{Tudi v predstavitvah lahko pišemo izreke in dokaze}
	% Naloga 2.3.2:
	% Oblikujte okolje itemize, tako da se bo njegova vsebina postopoma odkrivala.
	% Ne smete uporabiti ukaza `pause'.
	% Beseda `največje' naj bo poudarjena šele na četrti podprosojnici.
	\begin{itemize}
		\item<1-> Naj bo $p$ \uncover<4->{\alert{največje}} praštevilo.
		\item<2-> Naj bo $q$ produkt števil $1$, $2$, \ldots, $p$.
		\item<3-> Število $q+1$ ni deljivo z nobenim praštevilom, torej je $q+1$ praštevilo.
		\item<4-> To je protislovje, saj je $q+1>p$. \qedhere
	\end{itemize}
	\begin{izrek}
	   $Praštevil je neskončno mnogo.$
	\end{izrek}
	\begin{proof}
	   Dokaz je preprost.	
	\end{proof}
 \end{frame}
 
