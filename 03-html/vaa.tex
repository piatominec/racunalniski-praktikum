\documentclass{article} % Osnovna vrsta dokumenta

% Paketi za dodatne funkcije
\usepackage[utf8]{inputenc} % Podpora za UTF-8 znake
\usepackage[T1]{fontenc} % Pravilna kodirna shema za izhodne datoteke
\usepackage{amsmath} % Paket za matematične formule

\title{Osnovni dokument v LaTeXu}
\author{Vaše ime}
\date{\today} % Trenutni datum

\begin{document}
\author{UrosT}

% Glava dokumenta (naslov, avtor, datum)
\maketitle

\maketitle % Glava dokumenta (naslov, avtor, datum)
\begin{figure}[h!]
    \centering
    \includegraphics[width=0.5\textwidth]{path/to/your/image.jpg}
    \caption{Opis slike}
    \label{fig:sample-image}
\end{figure}
\section{Uvod}
To je osnovni LaTeX dokument. Tukaj prikazujemo, kako lahko oblikujemo besedilo, uporabimo matematične formule ter vključimo več razdelkov v dokumentu.

\section{Matematične formule}
LaTeX je odličen za prikaz matematičnih izrazov. Na primer, kvadratna enačba je:
\[
ax^2 + bx + c = 0
\]
Rešimo jo z uporabo formule:
\[
x = \frac{-b \pm \sqrt{b^2 - 4ac}}{2a}
\]

\section{Seznami}
Prav tako lahko dodamo sezname:
\subsection{Oštevilčen seznam}
\begin{enumerate}
    \item Prva točka
    \item Druga točka
    \item Tretja točka
\end{enumerate}

\subsection{Neoštevilčen seznam}
\begin{itemize}
    \item Prva postavka
    \item Druga postavka
    \item Tretja postavka
\end{itemize}

\section{Zaključek}
To je kratek primer LaTeX dokumenta. Z uporabo različnih paketov in ukazov lahko oblikujemo profesionalne dokumente.

\end{document}
